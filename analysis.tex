% Options for packages loaded elsewhere
\PassOptionsToPackage{unicode}{hyperref}
\PassOptionsToPackage{hyphens}{url}
%
\documentclass[
]{article}
\usepackage{lmodern}
\usepackage{amssymb,amsmath}
\usepackage{ifxetex,ifluatex}
\ifnum 0\ifxetex 1\fi\ifluatex 1\fi=0 % if pdftex
  \usepackage[T1]{fontenc}
  \usepackage[utf8]{inputenc}
  \usepackage{textcomp} % provide euro and other symbols
\else % if luatex or xetex
  \usepackage{unicode-math}
  \defaultfontfeatures{Scale=MatchLowercase}
  \defaultfontfeatures[\rmfamily]{Ligatures=TeX,Scale=1}
\fi
% Use upquote if available, for straight quotes in verbatim environments
\IfFileExists{upquote.sty}{\usepackage{upquote}}{}
\IfFileExists{microtype.sty}{% use microtype if available
  \usepackage[]{microtype}
  \UseMicrotypeSet[protrusion]{basicmath} % disable protrusion for tt fonts
}{}
\makeatletter
\@ifundefined{KOMAClassName}{% if non-KOMA class
  \IfFileExists{parskip.sty}{%
    \usepackage{parskip}
  }{% else
    \setlength{\parindent}{0pt}
    \setlength{\parskip}{6pt plus 2pt minus 1pt}}
}{% if KOMA class
  \KOMAoptions{parskip=half}}
\makeatother
\usepackage{xcolor}
\IfFileExists{xurl.sty}{\usepackage{xurl}}{} % add URL line breaks if available
\IfFileExists{bookmark.sty}{\usepackage{bookmark}}{\usepackage{hyperref}}
\hypersetup{
  hidelinks,
  pdfcreator={LaTeX via pandoc}}
\urlstyle{same} % disable monospaced font for URLs
\usepackage{longtable,booktabs}
% Correct order of tables after \paragraph or \subparagraph
\usepackage{etoolbox}
\makeatletter
\patchcmd\longtable{\par}{\if@noskipsec\mbox{}\fi\par}{}{}
\makeatother
% Allow footnotes in longtable head/foot
\IfFileExists{footnotehyper.sty}{\usepackage{footnotehyper}}{\usepackage{footnote}}
\makesavenoteenv{longtable}
\setlength{\emergencystretch}{3em} % prevent overfull lines
\providecommand{\tightlist}{%
  \setlength{\itemsep}{0pt}\setlength{\parskip}{0pt}}
\setcounter{secnumdepth}{-\maxdimen} % remove section numbering

\author{}
\date{}

\begin{document}

\[
\newcommand{\R}{\mathbb{R}}
\newcommand{\com}{\mathbb{C}}
\newcommand{\N}{\mathbb{N}}
\newcommand{\Z}{\mathbb{Z}}
\newcommand{\e}{\varepsilon}
\newcommand{\sequence}[1]{(#1_n)_{n=1}^{\infty}}
\newcommand{\braces}[1]{\left\{#1\right\}}
\newcommand{\ra}{\rightarrow}
\newcommand{\pdiff}[2]{\frac{\partial #1}{\partial #2}}
\newcommand{\diff}[3]{\frac{d^{#3}#1}{d#2^{#3}}}
\newcommand{\summ}[2]{\sum_{#1}^{#2}}
\newcommand{\fbraks}[1]{\left(#1\right)}
\newcommand{\tbraks}[1]{\left[#1\right]}
\]

\hypertarget{gamma-and-beta-function}{%
\section{Gamma and Beta function}\label{gamma-and-beta-function}}

\[\Gamma(z) = \int_0^\infty t^{z-1}e^{-t}dt\]
\[\beta(x,y) = \int_0 ^1 t^{x-1}(1-t)^{y-1}dt = 2\int_0 ^{\frac{\pi}{2}} \sin^{2x-1}{\theta} \cos^{2y-1}{\theta} d\theta\]

\begin{longtable}[]{@{}ll@{}}
\toprule
\begin{minipage}[b]{0.53\columnwidth}\raggedright
Column 1\strut
\end{minipage} & \begin{minipage}[b]{0.41\columnwidth}\raggedright
Column 2\strut
\end{minipage}\tabularnewline
\midrule
\endhead
\begin{minipage}[t]{0.53\columnwidth}\raggedright
\(\Gamma(z) = (z-1)\Gamma(z-1)\)\strut
\end{minipage} & \begin{minipage}[t]{0.41\columnwidth}\raggedright
\(\beta(x,y)= \frac{\Gamma(x)\Gamma(y)}{\Gamma(x+y)}\)\strut
\end{minipage}\tabularnewline
\begin{minipage}[t]{0.53\columnwidth}\raggedright
\(\Gamma(n) = (n-1)!\) \(,\forall n\in \mathbb{N}\)\strut
\end{minipage} & \begin{minipage}[t]{0.41\columnwidth}\raggedright
Blank\strut
\end{minipage}\tabularnewline
\begin{minipage}[t]{0.53\columnwidth}\raggedright
\(\Gamma(\frac{1}{2}) = \sqrt{\pi}\)\strut
\end{minipage} & \begin{minipage}[t]{0.41\columnwidth}\raggedright
Blank\strut
\end{minipage}\tabularnewline
\begin{minipage}[t]{0.53\columnwidth}\raggedright
\(\Gamma(n)\Gamma(n+\frac{1}{2}) = 2^{1-2n}\sqrt{\pi}\cdot\Gamma(2n)\)\strut
\end{minipage} & \begin{minipage}[t]{0.41\columnwidth}\raggedright
Blank\strut
\end{minipage}\tabularnewline
\begin{minipage}[t]{0.53\columnwidth}\raggedright
\(\Gamma(1-n)\Gamma(n)= \frac{\pi}{\sin{\pi n}}\)\strut
\end{minipage} & \begin{minipage}[t]{0.41\columnwidth}\raggedright
Blank\strut
\end{minipage}\tabularnewline
\bottomrule
\end{longtable}

\hypertarget{sequences}{%
\section{Sequences}\label{sequences}}

\hypertarget{defnitions}{%
\subsection{Defnitions}\label{defnitions}}

\begin{enumerate}
\def\labelenumi{\arabic{enumi}.}
\item
  A sequence of real number is a function,\(f:\N \ra \R\) . Commonly
  represented as \(\sequence{a} = \braces{a_1, a_2, a_3, \cdots}\).
\item
  A series is a summation of the sequnce elements. Series of above
  sequence is written as:- \[
  A = \sum_{n \in \N} a_n=\sum_{n=1}^{\infty} a_n = a_1+a_2+a_3+\cdots
  \]
\item
  Partial sum of a series can be written as :
  \[A_n = \sum_{k=1}^{n}a_k\] Note this is a finite sum of the orginal
  series \(A\).
\end{enumerate}

\hypertarget{convergence-of-sequence}{%
\subsection{Convergence of sequence}\label{convergence-of-sequence}}

Given, \(\sequence{a}\) in \(\R\). \(a\in \R\). The sequence
\(\sequence{a}\) converges to \(a\) if only if
\[\forall \e>0 \exists N \in \N \text{ s.t } n>N \implies|x_n-x|<\e\].

Syntactically, \[\lim a_n = \lim_{n\rightarrow \infty}a_n = a\]

\hypertarget{properties}{%
\subsubsection{Properties}\label{properties}}

Say, \(\sequence{s}\ra s\) and \(\sequence{t}\ra t\).

\begin{itemize}
\tightlist
\item
  \(\lim s_n + t_n = s+t\)
\item
  \(\lim c\cdot s_n = c\cdot s\) for all \(c \in R\)
\item
  \(\lim c+ s_n = c+ s\) for all \(c \in R\)
\item
  \(\lim t_n s_n = ts\)
\item
  \(\lim \frac{1}{s_n} = \frac{1}{s}\) {[}considering all is
  well-defined{]}
\end{itemize}

\hypertarget{sepcial-sequence}{%
\subsubsection{Sepcial sequence}\label{sepcial-sequence}}

\begin{itemize}
\tightlist
\item
  if \(p>0\) then \(\lim \frac{1}{n^p} = 0\)
\item
  if \(p>0\) then \(\lim p^{\frac{1}{n}} = 1\)
\item
  \(\lim n^{\frac{1}{n}} = 1\)
\item
  if \(p>0\) and \(a \in \R\) then \(\lim \frac{n^a}{(1+p)^n} = 0\)
\item
  if \(|x|<1\) then \(\lim x^n = 0\)
\item
  \(\lim (1+n)^{1/n} = 1\)
\end{itemize}

\hypertarget{convergence-of-series}{%
\subsection{Convergence of Series}\label{convergence-of-series}}

Let, \[A = \sum_n a_n\] \[ A_n= \sum_{k=1}^{n} a_k\] See carefully that,
series \(A\) 's convergence is same as convergence of the following
sequence of partial sums: \(\braces{A_1, A_2, A_3...}\). So more
formally one can write, \[\lim_{n \ra \infty} A_n = A\]

So all the techniques used in convergence of sequence can be used here
too.

\hypertarget{properties-1}{%
\subsection{Properties}\label{properties-1}}

\[\R\]

\end{document}
